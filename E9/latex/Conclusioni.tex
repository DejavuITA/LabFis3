\section{Conclusioni}
Siamo riusciti a costruire una sorgente di corrente ideale indipendente dal carico.
Abbiamo anche osservato che, se il valore del carico diventa troppo grande, il transistor entra in saturazione e a quel punto a determinare la corrente di collettore non è più la corrente di base ma il carico stesso.

Siamo riusciti a progettare e creare un amplificatore ad emettitore comune, dimensionando le resistenze in modo da ottenere un guadagno $G \approx$ -10 alla frequenza di \SI{1}{\kilo\hertz} e una corrente di quiescenza di \SI{1}{\milli\ampere}.
Ne abbiamo inoltre osservato il comportamento al variare della tensione $V_{pp}$ evidenziando un effetto di ``clipping'' sia sulla semionda negativa che su quella positiva. 
Infine è stata analizzata la caratteristica in uscita del transistor utilizzato durante l'esperienza.
È stato possibile apprezzare la differenza tra regione di saturazione (la corrente di collettore dipende dalla tensione $V_{CE}$) e regione attiva (la corrente di collettore è determinata dalla corrente di base).
Infine, dai dati  raccolti è stato stimato il guadagno in corrente del transistor. Il valore ottenuto è compatibile con quello stimato nell'esperienza precedente ($\beta = 307.23 \pm 0.03 $).
