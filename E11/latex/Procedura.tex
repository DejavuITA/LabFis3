\section{Amplificatore invertente e non-invertente}
Il primo circuito da noi analizzato consiste in un amplificatore invertente accoppiato DC. Lo schema è riportato in Fig.(??). Come vediamo, il nostro amplificatore operazionale è collegato con un circuito di \textbf{feedback} \textbf{negativo} (ovvero viene portato un po' del segnale in output all'ingresso invertente). Così facendo possiamo avere un controllo sul segnale in uscita che altrimenti sarebbe, per come è costruito l'op-amp, $\pm V$ (avendo un guadagno di $10^6$). Analizziamo dunque il circuito utilizzato. Se consideriamo il nostro amplificatore operazionale ideale, abbiamo che $\Delta V_{12}=0$ (prima condizione di idealità). Il punto $A$ sarà un ground virtuale. Possiamo dunque imporre $I_1=\frac{V_{in}-V_A}{R_1}=\frac{V_{in}}{R_1}$.  Inoltre $I_2=\frac{V_A-V_{out}}{R_2}=\frac{-V_{out}}{R_2}$.
Sfruttando la seconda condizione di idealità, $\Delta I_{12}=0$, otteniamo $V_{out}=-\frac{R_2}{R_1} V_{in}$. Il guadagno del nostro circuito amplificatore sarà dunque:

$$G=-\frac{R_2}{R_1}$$


Esso è negativo in quanto sfasato rispetto al segnale in ingresso di $\pi$.
La richiesta fatta era di ottenere un guadagno di circa -10. Abbiamo dunque scelto di usare $R_1=1001.6\pm0.3$ e $R_2=9987.1\pm0.3$. Il circuito è stato alimentato con un segnale in input sinusoidale alla frequenza di 1kHz. Per valori picco-picco maggiori di $3V$ abbiamo notato l'ormai classico effetto di clapping del segnale, in quanto la tensione in output raggiungeva il valore massimo fornito dalla polarizzazione DC dell'op-amp. 

Ne abbiamo inoltre analizzato l'andamento al variare della frequenza. Come già accaduto per l'amplificatore alle differenze, abbiamo notato che a frequenze elevato il guadagno diminuiva considerevolmente, con anche uno sfasamento rilevante dei segnali. In Fig.(??) è riportato un grafico del guadagno in funzione della frequenza. 

$$Grafico??$$

$$Se abbiamo i dati$$


Crediamo che il motivo di tale smorzamento del segnale sia la presenza dei transistor e delle capacità nell'amplificatore operazionale.


Successivamente abbiamo montato l'amplificatore non invertende come mostrato in Fig.(??). Con gli stessi ragionamenti fatti sopra, possiamo calcolare il guadagno di tale circuito: $V_A=V_{in}=V_{out}\frac{R_1}{R_2+R_1} \Rightarrow V_{out}=(1+\frac{R_2}{R_1})$. Il guadagno, positivo in questo caso, risulta essere: 

$$G=1+\frac{R_2}{R_1}$$





\section{Sommatore}


\section{Integratore}


\section{Derivatore}