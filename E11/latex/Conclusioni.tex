\section{Conclusioni}

Abbiamo analizzato diversi circuiti costruiti utilizzando un'amplificatore operazionale. Nel circuito amplificatore ne abbiamo apprezzato la comodità in quanto non abbiamo bisogno di dimensionare partitori e correnti di quiescenza come nel caso dell'amplificatore alle differenze con transistor BJT. 

Ne abbiamo studiato l'uso in circuiti sommatori, integratori e derivatori.
In questi ultimi due la caratteristica più interessante è l'indipendenza, almeno in prima approssimazione, dalla frequenza.
Sappiamo infatti che l'op-amp non funziona più sopra una determinata frequenza.
Tuttavia, rispetto ai tradizionali circuiti derivatori e integratori studiati precedentemente (filtri passa alto/basso con l'approssimazione di C ed R piccole), il range di frequenze risulta molto più ampio.
Abbiamo anche affrontato il problema della non idealità dell'op-amp e della necessità di creare dei circuiti che tengano conto di ciò (resistenza di feedback dell'integratore).