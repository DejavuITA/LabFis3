\section{Conclusioni}

In questa esperienza siamo riusciti ad assemblare un amplificatore alle differenze con un guadagno differenziale $G_{diff} \,=\, 30$ e un guadagno in modo comune $\left| G_{CM} \right| \,\leq\, 1$.
Grazie alle formule che legano i diversi guadagni abbiamo inoltre calcolato il fattore di reiezione a modo comune $CMRR$ e il valore della resistenza intrinseca di emettitore $r_e$.
Aggiungendo una sorgente di corrente continua a valle del circuito ne abbiamo migliorato le caratteristiche: in particolare abbiamo diminuito di tre ordini di grandezza il guadagno in modo comune $G_{CM}$ e di conseguenza aumentato di tre ordini di grandezza il fattore di reiezione a modo comune $CMRR$.