\section{Amplificatore alle differenze}
Un amplificatore alle differenze è un circuito che amplifica la differenza di due segnali in ingresso. 
Il circuito da noi realizzato è rappresentato in Fig.().
Come al solito abbiamo dimensionato le resistenze in modo da ottenere i valori di corrente di quiescenza richiesta ($I_q=\SI{0.5}{\milli\ampere}$), un guadagno differenziale di $\approx 30$ e un guadagno a modo comune inferiore a 1.

Possiamo dunque imporre la seguente condizione:

\begin{equation}
V_B-V_E=2I_q R_1 + I_q R_E + 0.6
\label{eq:1}
\end{equation}

Tale equazione tiene conto della geometria del circuito. Come vediamo essa non permette tuttavia una calcolo diretto di $R_E$ ed $R_1$. Per fare ciò è necessario imporre la condizione di $G_{diff} \approx 30$:

\begin{equation}
G_{diff}=\frac{R_C}{2(R_E+r_e)}
\label{eq:2}
\end{equation}

Con $r_e$ resistenza intrinseca dell'emettitore. Come sappiamo dallo studio del modello $Early$, essa può essere stimata dall'equazione sperimentale $r_e = [\frac{25}{I_q}] \Omega$. Imponendo una corrente di quiescenza di \SI{0.5}{\milli\ampere} otteniamo immediatamente $r_e=50 \Omega$. 

Infine, avendo utilizzato una $R_C=(9.972\pm0.002)\si{\kilo\ohm}$, è possibile attraverso eq.(\ref{eq:1}) e eq.(\ref{eq:2}) stimare i valori adeguati di $R_1$ ed $R_E$. 

Come consigliato in laboratorio, sono state utilizzate una $R_E=(120\pm6) \Omega$\footnote{In questo caso abbiamo scelto di tenere il valore nominale fornito dal costruttore così da avere delle semplificazioni nel calcolo degli errori sui guadagni} e una $R_1= (9.933 \pm 0.002)\si{\kilo\ohm}$.

Con questi valori di resistenza la corrente di quiescenza $I_q$ è di \SI{0.7}{\milli\ampere} (sperimentalmente abbiamo misurato una corrente di $(0.719 \pm 0.003) \si{\milli\ampere}$.

Per rendere l'analisi dei segnali più semplice, abbiamo collegato il $V_{in}2$ a terra.
Così facendo abbiamo ottenuto solo un'amplificazione del segnale in ingresso a $V_{in}1$. 

Riportiamo ora un grafico dei dati sperimentali acquisiti. 

$$****grafico****$$


Il guadagno è stato stimato dai valori picco-picco del segnale in ingresso e di quello in input:

$$G_{diff}=\frac{V_{out}}{V_{in}}$$

Sperimentalmente abbiamo ottenuto il valore $G_{diff}=30.09 \pm 0.01$


Utilizzando eq.(\ref{eq:2}) è possibile calcolare il valore di resistenza intrinseca dell'emettitore: $r_e=\frac{R_c-2R_E G_{diff}}{2G_{diff}}$.
Sfruttando la composizione degli errori, abbiamo ottenuto il valore $r_e=$


\section{Amplificatore alle differenze con sorgente di corrente}