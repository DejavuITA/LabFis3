Progettare e studiare il funzionamento di amplificatore alle differenze. Successivamente sarà analizzato un amplificatore alle differenze con sorgente di corrente costante. 

\section{Strumenti}
%
%\phantom{porcodio!}
\noindent
\begin{minipage}{.5\linewidth}
$\bullet \quad$3 Transistor BC107B\\
$\bullet \quad$Generatore di forme d'onda (Agilent 33120A)\\
\phantom{xxxx}frequency range: \SI{100}{\micro\hertz} - \SI{15}{\mega\hertz}\\
$\bullet \quad$Generatore DC (Agilent E3631A)\\
\phantom{xxxx}$0-6\,\si{\volt}$, $\SI{5}{\ampere}$ / $0-\pm25\,\si{\volt}$, $\SI{1}{\ampere}$\\
\end{minipage}%
\begin{minipage}{.5\linewidth}
$\bullet \quad$Cablaggio, breadboard e resistenze varie\\
$\bullet \quad$Multimetro digitale (Agilent 34410A)\\
$\bullet \quad$Oscilloscopio (Agilent DSO-X 2002A)\\
\phantom{xxxx}\SI{70}{\mega\hertz} bandwidth, 2GSa/s sample rate\\
\end{minipage}