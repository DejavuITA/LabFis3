\section{Diodo semplice: caratteristica volt-amperometrica}

Per studiare la caratteristica volt-amperometrica del diodo abbiamo dovuto separare l'analisi in due fasi distinte: una prima fase in cui abbiamo analizzato il comportamento del diodo polarizzato in diretta e una seconda fase in cui ne abbiamo analizzato il comportamento in inversa.
Utilizzando la breadboard come supporto abbiamo collegato il diodo al genertore di tensione continua (max $\pm \SI{25}{\volt}$) %e messo un amperometro in serie
e abbiamo progressivamente aumentato la tensione osservando sull'amperometro collegato in serie la corrente che GIRAVA nel circuito.
Una piccola accortezza --> ID <= 700 mA nel diodo

\begin{equation}
I_{D} \, = \, I_{S} \left( e^{\frac{q V_d}{nKT}} -1 \right)
\label{eq:diode}
\end{equation}

\section{Cella solare: caratteristica volt-amperometrica}
\subsection{cella solare al buio}

\begin{equation}
FF \, = \, \frac{I_{@P_{max}} \,\, V_{@P_{max}}}{I_{sc} \,\, V_{oc}}
\label{eq:FF}
\end{equation}

\subsection{cella solare alla luce}


\section{ponte di Graetz}