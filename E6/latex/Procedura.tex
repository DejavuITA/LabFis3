\section{Diodo 1N4007: caratteristica volt-amperometrica}

\begin{wrapfigure}[22]{r}[0pt]{130mm}
	\caption{ciao}
	\label{fig:diodo}
	\includegraphics[width=0.75\textwidth]{diodo.pdf}
\end{wrapfigure}

Per studiare la caratteristica volt-amperometrica del diodo abbiamo dovuto separare l'analisi in due fasi distinte: una prima fase in cui abbiamo analizzato il comportamento del diodo polarizzato in diretta e una seconda fase in cui ne abbiamo analizzato il comportamento in inversa.
Utilizzando la breadboard come support,o abbiamo creato un circuito composto dal genertore di tensione continua (max $\pm \SI{25}{\volt}$), il diodo e il multimetro digitale in modalità amperometro. Aumentando progressivamente la tensione, abbiamo letto sull'amperometro i valori della corrente che attraversava il circuito.
Facendo attenzione che la corrente non superasse il valore di \SI{700}{\milli\ampere} abbiamo quindi popolato l'asse positivo del grafico. In seguito abbiamo girato il diodo in modo che fosse alimentato in inversa e abbiamo popolato anche l'asse negativo del grafico, ottenendo la caratteristica volt-amperometrica completa del diodo. Il risultato è esposto in Figura \ref{fig:diodo}.
\\
\\
Dal grafico osserviamo che i dati ricavati seguono la legge teorica, la cui formula è:
\begin{equation}
I_{D} \, = \, I_{S} \left( e^{\frac{q V_d}{nKT}} -1 \right)
\label{eq:diode}
\end{equation}

\section{Cella solare: caratteristica volt-amperometrica}

Abbiamo studiato la caratteristica volt-amperomentrica di una cella solare lavorando allo stesso modo di come abbiamo fatto per il diodo.
%\subsection{cella solare al buio}
La prima analisi sul comportamento della cella è stata svolta tenendo la cella solare al buio nella sua scatola, mentre una seconda analisi è stata completata con la cella solare sottoposta alla radiazione di una lampada da tavolo, facendo attenzione che quest'ultima non variasse significativamente di intensità.
Durante questa parte dell'esperienza abbiamo avuto cura che la corrente massima da cui era attraversata la cella non superasse i \SI{100}{\milli\ampere}. I dati ottenuti sono graficati in Figura \ref{fig:cella}.

\begin{wrapfigure}[22]{r}[0pt]{130mm}
	\includegraphics[width=0.75\textwidth]{cella.pdf}
	\caption{ciao}
	\label{fig:cella}
\end{wrapfigure}


\begin{equation}
FF \, = \, \frac{I_{@P_{max}} \,\, V_{@P_{max}}}{I_{sc} \,\, V_{oc}}
\label{eq:FF}
\end{equation}

%\subsection{cella solare alla luce}


\section{ponte di Graetz}