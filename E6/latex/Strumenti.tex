Osservare le caratteristice volt-amperometriche di un diodo e di una cella solare. Verificare il funzionamento di un ponte di Graetz.

\section{Strumenti}
%
\phantom{porcodio!}
\noindent
\begin{minipage}{.5\linewidth}
$\bullet \quad$Generatore di tensione alternata ($\SI{7.5}{\volt}_{rms}$)\\
$\bullet \quad$Cablaggio, breadboard e resistenza da $\SI{10}{\kilo\ohm}$\\
$\bullet \quad$Generatore di tensione continua ($max \SI{50}{\volt}$)\\
$\bullet \quad$Decade di condensatori
\end{minipage}%
\begin{minipage}{.5\linewidth}
$\bullet \quad$Oscilloscopio (Agilent DSO-X 2002A)\\
\phantom{xxxx}\SI{70}{\mega\hertz} bandwidth, 2GSa/s sample rate\\
$\bullet \quad$Multimetro digitale (Agilent 34410A)\\
$\bullet \quad$4 diodi MTSAKDA e cella solare
%$\bullet \quad$Generatore di forme d'onda (Agilent 33120A)\\
%\phantom{xxxx}frequency range: \SI{100}{\micro\hertz} - \SI{15}{\mega\hertz}\\
\end{minipage}