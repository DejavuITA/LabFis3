\section{Conclusioni}
Siamo riusciti ad analizzare la caratteristica V-I di un diodo e di una cella fotovoltaica. Abbiamo notato una leggera differenza nella forma dei due: infatti mentre il diodo polarizzato in inversa fino a $\SI{-50}{\volt}$ veniva attraversato da una corrente molto piccola ($\SI{12}{\nano\ampere}$ per la precisione), la cella fotovoltaica era attraversata da una corrente di ordini di grandezza superiore già a tensioni basse. Nello specifico già a $\SI{-2}{\volt}$ l'intensità di corrente da cui era attraversata la cella solare era di $\SI{80}{\nano\ampere}$, mentre a $\SI{-50}{\volt}$ la cella era attraversata da $\SI{30.20}{\milli\ampere}$.

Nella seconda parte dell'esperienza abbiamo analizzato, utilizzando l'oscilloscopio, un ponte di Graetz come raddrizzatore di segnale. In un secondo momento abbiamo visto come, aggiungendo un condensatore in parallelo al carico, il circuito possa essere utilizzato come sorgente di tensione continua, sebbene ci sia sempre una oscillazione del segnale con frequenza doppia rispetto al segnale in ingresso. Per minimizzare tale effetto (\emph{ripple}) è conveniente adoperare un condensatore quanto più grande si può.
Poichè, come già accennato, un capo del ponte sarà sempre a tensione maggiore rispetto all'altro, potrebbe essere conveniente utilizzare dei condensatori elettrolitici, che permettono di raggiungere capacità maggiori di quelle ottenibili con i classici condensatori a dielettrico solido, a parità di volume occupato.