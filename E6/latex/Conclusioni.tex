\section{Conclusioni}
Siamo riusciti ad analizzare la caratteristica V-I di un diodo e di una cella fotovoltaica. Utilizzando l'oscilloscopio abbiamo analizzato un ponte di Graetz come raddrizzatore di segnale. Abbinando un condensatore abbiamo visto come esso possa essere utilizzato come sorgente di tensione continua, sebbene ci sia sempre una oscillazione del segnale con frequenza doppia rispetto al segnale in ingresso (ripple). Per minimizzare tale effetto è conveniente adoperare un condensatore quanto più grande si può. Poichè, come già accennato, un capo del ponte sarà sempre a tensione maggiore rispetto all'altra, potrebbe essere conveniente utilizzare dei condensatori elettrolitici, che permettono di raggiungere capacità molto maggiori di quelle ottenibili con i classici condensatori. 