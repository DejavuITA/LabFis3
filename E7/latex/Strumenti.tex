Studiare il funzionamento di un rivelatore di fronti d'onda, analizzare la caratteristica V-I di un diodo Zener e suo funzionamento come stabilizzatore di tensione

\section{Strumenti}
%
\phantom{porcodio!}
\noindent
\begin{minipage}{.5\linewidth}
$\bullet \quad$Generatore di tensione continua ($DC \pm \SI{25}{\volt}_{rms}$)\\
$\bullet \quad$Cablaggio, breadboard, resistenza da $\SI{1}{\kilo\ohm}$\\
$\bullet \quad$ Diodo 1N4007 e diodo Zener BZX85C6V8\\
$\bullet \quad$Decade di resistenze
\end{minipage}%
\begin{minipage}{.5\linewidth}
$\bullet \quad$Decade di condensatori\\
$\bullet \quad$Oscilloscopio (Agilent DSO-X 2002A)\\
\phantom{xxxx}\SI{70}{\mega\hertz} bandwidth, 2GSa/s sample rate\\
$\bullet \quad$Multimetro digitale (Agilent 34410A)\\
%$\bullet \quad$Generatore di forme d'onda (Agilent 33120A)\\
%\phantom{xxxx}frequency range: \SI{100}{\micro\hertz} - \SI{15}{\mega\hertz}\\
\end{minipage}
