\section{Conclusioni}
Si vede da figura (\ref{fig:resistenze}) che per resistenze piccole ($\approx$ 1 $\Omega$) la misura effettuata con il circuito ``a valle'' risulta essere, se non si applicano correzioni, non compatibile con il valore reale di resistenza. E' facile capire il motivo di ciò analizzando il circuito: ricordando l'equazione (\ref{brugnagay}), vediamo subito che $\frac{V}{I}=R_X+R_A$. Dunque, se $R_A$ ed $R_X$ sono confrontabili tra loro, è evidente che il valore stimato sarà completamente sbagliato. Analogamente, si vede che il circuito ``a monte'' non fornisce un valore attendibile di $R_X$ per resistenze grandi ($\approx$ 1 $M\Omega$). Ciò si può facilmente vedere dal'equazione (\ref{casapagliaccio}). L'amperometro sentirà una resistenza a valle pari a $R_{eq}=(\frac{1}{R_V}+\frac{1}{R_X})^{-1}$. E' evidente che, nel momento in cui $R_X \approx R_V$ e addirittura $R_X > R_V$, la resistenza del tratto circuitale comprendente voltmetro e resistenza in esame sarà minore di $R_X$. Anche in questi casi risulterà quindi necessario applicare la correzione del caso. 

Per valori di resistenza intermedi ($1000-100000 \Omega$) entrambi i circuiti si dimostrano validi e anche senza applicare correzioni si può ottenere una buona stima di $R_X$.

Applicando la correzione abbiamo notato che non vi sono differenze apprezzabili tra i due circuiti utilizzati per ricavare il valore incognito di resistenza. E' possibile ottenere con ambedue i sistemi un valore di $R_X$ affetto da errore percentuale anche solo del $2\%$ nei casi migliori. Ricordiamo infatti che i dati affetti da errori percentuali più grandi sono quelli in cui non siamo riusciti ad ottimizzare il fondo scala ( non sempre si è riusciti a tenere gli indicatori degli strumenti oltre la metà quadrante). %Potrebbe essere interessante notare che il valore misurato con amperometro ``a monte'' per la resistenza di 1 $M \Omega$ risulta affetto da un errore percentuale consistente. Durante l'esperienza abbiamo effettuato questa misura con una tensione di 7 V, mentre la relativa misura ``a valle'' con tensione di 40 V. Non possiamo dire con certezza che causa di tale errore sia la diversa tensione applicata ma, visto che per le altre resistenze abbiamo tenuto praticamente invariata la tensione eseguendo le misure "monte" e "valle" e per ciascuna coppia gli errori sono circa uguali con entrambi i metodi, ciò potrebbe essere indice che lo strumento funziona in modo più preciso quando sia tensione che corrente che scorre nel circuito sono più alte. ***CREDO SIA UNA CAZZATA QUESTA, E' COLPA DEL FONDOSCALA DI MMIERDA***


 Abbiamo anche visto il comportamento di un carico resistivo non ohmico. E' interessante notare come il cambio scala è stato il fattore determinante sugli errori sulle misure. Nel grafico Volt-Amperometrico si vede infatti chiaramente che nel momento in cui si è passati dalla scala $10V$ a quella $50V$ gli errori sono cresciuti di 5 volte.
 
Siamo riusciti ad ipotizzare una legge che lega tensione e corrente. Non sappiamo se tale legge sia corretta, tuttavia possiamo dire che nel range di valori dove sono compresi i dati essa è verificata. Il valore di $b<1$ implica inoltre che aumentando la tensione la curva tende ad appiattirsi, ovvero la resistenza aumenta. 