\section{Conclusioni}
Come si evince dai grafici sopra riportati, non siamo riusciti ad individuare in quali casi sia più opportuno utilizzare la configurazione a "monte" o quella a "valle". Ciò potrebbe essere dovuto al fatto che non sempre è stato possibile effettuare le misure in condizioni ottimali. Infatti, per alcuni dati, è risultato impossibile avere le lancette degli strumenti oltre la metà quadrante. Tale limitazione è stata principalmente causata dal vincolo di potenza massima fornita alla resistenza. Dal grafico che riporta gli errori percentuali sui valori calcolati notiamo che le misure effettuate sono compatibili tra loro. Riteniamo che gli errori percentuali più significativi che si notano nel grafico siano da attribuire, come appena detto, al problema del connubio fondoscala-voltaggio massimo. Abbiamo anche visto il comportamento di un carico resistivo non ohmico. E' interessante notare come il cambio scala sia stato il fattore determinante nella stima degli errori sulle misure. Nel grafico Volt-Amperometrico si vede infatti chiaramente che nel momento in cui si è passati dalla scala $10V$ a quella $50V$ gli errori sono cresciuti di 5 volte. Ciò era stato previsto, ma si evince da ciò anche che lo strumento non possiede una sensibilità sufficiente per stimare errori statistici sulle misure.