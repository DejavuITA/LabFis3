\section{Strumenti}
\begin{wrapfigure}{r}{100mm}
    \centering
    \begin{subfigure}[h]{0.3\textwidth}
        \includegraphics[width=\textwidth]{R_monte.pdf}
        \caption{Schema del cricuito con amperometro a monte}
        \label{fig:monte}
    \end{subfigure}%
    ~ %add desired spacing between images, e. g. ~, \quad, \qquad etc.
      %(or a blank line to force the subfigure onto a new line)
    \begin{subfigure}[h]{0.3\textwidth}
        \includegraphics[width=\textwidth]{R_valle.pdf}
        \caption{Schema del cricuito con amperometro a valle}
        \label{fig:valle}
    \end{subfigure}%
    \caption{blah blah blah}
    \label{fig:circuiti}
\end{wrapfigure}

\begin{itemize} [noitemsep]
	\item 7 resistenze (1, 10, 100, 1k, 10k, 100k, 1M$\Omega$ ); 
	\item Cablaggio;
	\item 2 tester ICE;
	\item Generatore di corrente DC;
	\item Multimetro digitale;
	\item Breadboard o basetta sperimentale;
\end{itemize}