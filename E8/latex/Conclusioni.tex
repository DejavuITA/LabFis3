\section{Conclusioni}

Nella prima parte dell'esperienza abbiamo studiato il funzionamento di un transistor come interruttore di corrente.
Variando la corrente di base nel range $0 - 39\,\si{\micro\ampere}$, abbiamo notato che la corrente di collettore cresceva in modo direttamente proporzionale ($I_C \simeq \beta I_B$).
Nel range di corrente $\geq 40 \,\si{\micro\ampere}$, invece, abbiamo osservato che il transistor entrava in saturazione e non ci permetteva più di controllare la corrente di collettore con la corrente di base.

Abbiamo poi osservato il funzionamento del transistor come interruttore veloce, variando il segnale di entrata in frequenza grazie al generatore di forme d'onda.

Nella seconda parte dell'esperienza abbiamo osservato il comportamento di un transistor nell'utilizzo come emitter follower.
Abbiamo notato che, con il primo circuito analizzato, l'onda in uscita era tagliata nella semionda negativa.
Ciò è dovuto al fatto che, in presenza di segnale negativo, il transistor entra in stato di interdizione.
Per ovviare a questo fatto abbiamo polarizzato costantemente il transistor imponendo una d.d.p. con un secondo generatore posto a \SI{-12}{\volt}.

L'ultimo circuito assemblato è stato l'emitter follower con partitore.
Questo circuto rimediava al problema di taglio della semionda negativa del primo circuito, senza l'utilizzo di due generatori di tensione. Unico difetto di tale circuito è la dissipazione di potenza da parte delle resistenze del partitore.

Un problema comune a tutti gli ultimi tre circuiti è il ``clipping'': cioè il fenomeno in cui il segnale in uscita è limitato in modulo ad un certo valore.
Ciò avviene quando il segnale in entrata al circuito ha un'ampiezza tale da mandare in interdizione il transistor.