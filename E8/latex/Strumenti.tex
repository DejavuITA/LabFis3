Studiare il funzionamento di un transistor bipolare nei suoi utilizzi come interruttore %(circuiti 1 e 2)
e come emitter follower.% (circuiti 3, 4 e 5).

\section{Strumenti}
%
%\phantom{porcodio!}
\noindent
\begin{minipage}{.5\linewidth}
$\bullet \quad$Transistor BC107B\\
$\bullet \quad$Cablaggio, breadboard e decade di resistenze\\
$\bullet \quad$Un LED verde, resistenze e condensatori vari\\
$\bullet \quad$Generatore DC (Agilent E3631A)\\
\phantom{xxxx}$0-6\,\si{\volt}$, $\SI{5}{\ampere}$ / $0-\pm25\,\si{\volt}$, $\SI{1}{\ampere}$\\
$\bullet \quad$Multimetro digitale (Agilent 34410A)\\
\end{minipage}%
\begin{minipage}{.5\linewidth}
$\bullet \quad$Oscilloscopio (Agilent DSO-X 2002A)\\
\phantom{xxxx}\SI{70}{\mega\hertz} bandwidth, 2GSa/s sample rate\\
\phantom{xxxx}impedenza in entrata $Z_{in}=\SI{1}{\mega\ohm}$\\
$\bullet \quad$Generatore di forme d'onda (Agilent 33120A),\\
\phantom{xxxx}impedenza in uscita $Z_s=\SI{50}{\ohm}$\\
\phantom{xxxx}frequency range: \SI{100}{\micro\hertz} - \SI{15}{\mega\hertz}\\
\end{minipage}