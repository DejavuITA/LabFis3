\section{Studio di circuiti RC: R e C conosciute.}
\subsection{Procedura di acquisizione dati}
L'oscilloscopio permette di visualizzare su uno schermo la forma d'onda di un segnale. Noi cercheremo dunque di stimare graficamente la costante di tempo di un circuito RC. Per fare ciò consideriamo anzitutto l'equazione di un circuito RC:
\begin{equation}
V(t)=V_0(1-e^{-\frac{t}{\tau}})
\label{RC}
\end{equation}
con $\tau=RC$. Se fissiamo arbitrariamente un tempo $t=\tau$, è evidente che il valore di tensione raggiunto agli estremi del condensatore dovrà essere $V=V_0(1-e^{-1})$. Poichè è stato scelto di usare come voltaggio massimo $V_0 = 1\si{\volt}$ il valore di tensione raggiunto dal condensatore dopo un tempo $t=\tau$ sarà proprio $V=(1-e^{-1})V_0 \simeq 0.632\si{\volt}$. 
Per analizzare le forma d'onda mostrate a video abbiamo utilizzato dei cursori che ci fornivano le cordinate di alcuni punti a scelta sullo schermo. Posizionando un cursore nel punto in cui il voltaggio passava da $0\si{\volt}$ a $1\si{\volt}$ e un secondo cursore all'ordinata di valore $0.632\si{\volt}$ e all'ascissa di intersezione con la forma d'onda di risposta del circuito è stato possibile ricavare il valore $\Delta t = \tau_{exp}$.
****

***

***


Riportiamo ora i dati da noi rilevati.

\begin{center}\centerline{
\label{Tabella}
\begin{tabular}{|c|c|c|c|c|}
\hline
Resistenza [$\Omega$] & Capacità [$nF$] & $\tau_{exp}$ [$\mu s$] & $\tau_{teo}$ [$\mu s$]&frequenza [$Hz$]\\
\hline
$2992.4 \pm 1$ & $1.13 \pm 0.05$ & $3.698 \pm 0.005$ & $3.4 \pm 0.1$ & 10k \\
$3988.5 \pm 1$ & $73.2 \pm 0.3$ & $300 \pm 2 $ & $293 \pm 1$ & 100 \\
$3988.4 \pm 1$ & $226 \pm 2$ & $920 \pm 5$ & $903 \pm 8$ & 20 \\
\hline
\end{tabular}}
\end{center}