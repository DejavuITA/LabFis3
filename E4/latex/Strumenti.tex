%\begin{wrapfigure}[6]{r}[0pt]{100mm}
%	\centering
%    \includegraphics[width=0.40\textwidth]{circuito.pdf}
%    \caption{Schema del cricuito utilizzato}
%    \label{fig:circuito}
%\end{wrapfigure}

Utilizzare il ponte di Wien per determinare il valore di un capacitore incognito e verificare il funzionamento del ponte di Wien invertito come filtro notch.

\section{Strumenti}
$\bullet \quad$Oscilloscopio (Agilent DSO-X 2002A)\\
\phantom{xxxx}\SI{70}{\mega\hertz} bandwidth, 2GSa/s sample rate\\
$\bullet \quad$Cablaggio\\
$\bullet \quad$Breadboard\\
$\bullet \quad$Generatore di forme d'onda (Agilent 33120A)\\
\phantom{xxxx}frequency range: \SI{100}{\micro\hertz} - \SI{15}{\mega\hertz}\\
$\bullet \quad$Multimetro digitale (Agilent 34410A)\\
$\bullet \quad$Resistenze da \SI{1}{\kilo\ohm}, condensatori da \SI{0.1}{\micro\farad}\\
$\bullet \quad$Condensatore di capacità incognita\\
$\bullet \quad$Decade di resistenze 