\section{Acquisizione dati}
Una volta montato il circuito e collegato al generatore di forme d'onda sono stati collegati i due canali dell'oscilloscopio alle uscite del circuito (CH1 e CH2, come in Figura \ref{fig:circuito}). 
\`E stata fissata una tensione di $5\,Vpp$ e, dopo aver stabilizzato con il comando $trigger$ le forme d'onda sullo schermo e averle allineate, sono state impostate le funzioni integrate dell'oscilloscopio che permettevano di ottenere direttamente a schermo i valori di picco dei due segnali e di differenza di fase tra i medesimi.

%Per ricavare valori più stabili e meno affetti da rumore abbiamo impostato l'oscilloscopio in modo tale che calcolasse un valore medio da 64 misure.
%Ricordiamo inoltre che nei grafici le barre d'errore non sono apprezzabili in quanto sono %troppo piccole per la scala utilizzata.