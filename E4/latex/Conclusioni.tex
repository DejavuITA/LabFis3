\section{Conclusioni}

Attraverso il ponte di Wien siamo riusciti ad ottenere una misura precisa del valore di un capacitore incognito. Un valore compatibile con quello misurato dal multimetro e dichiarato dal costruttore è ottenibile sia dalle equazioni di bilanciamento che dalla loro combinazione. Notiamo dai dati riportati in tabella che quest'ultimo procedimento è quello che permette una stima affetta da minor errore. 
Abbiamo verificato il funzionamento del ponte di Wien come filtro notch. Esso, come già detto, non risulta efficace come un filtro notch costruito utilizzando capacitore in serie ad un'induttanza. Inoltre, per le scelte da noi fatte di resistenze e capacità, l'intensità picco-picco del segnale in output non potrà mai essere superiore a $\frac{V_{in}}{2}$. 