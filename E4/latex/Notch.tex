\section{Ponte di Wien invertito come filtro notch}
\subsection{Acquisizione e analisi dati}
Utilizzando un ponte di Wien ``invertito'' è possibile creare un filtro notch. Esso non è efficace come quello montato nell'esperienza precedente ma è tuttavia interessante in quanto entrambi i capi del circuito contribuiscono alla creazione del segnale finale. Come si vede in Fig. \ref{fig:Winv} sia $V_1$ che $V_2$ sono variabili nel tempo. Un'implicazione di ciò è che un eventuale circuito collegato al ponte di Wien non può essere messo a terra (per questo non è stato possibile eseguire direttamente una misura della differenza di potenziale tra i capi attraverso l'oscilloscopio, il quale è per struttura messo a terra).
\begin{wrapfigure}[17]{r}[0pt]{80mm}
	\centering
    \includegraphics[width=0.30\textwidth]{schema2.pdf}
    \caption{Schema del ponte di Wien invertito}
    \label{fig:Winv}
\end{wrapfigure}

\begin{wrapfigure}[30]{r}[0pt]{90mm}
%	\centering
    \includegraphics[width=90mm]{notch.pdf}
    \caption{Diagrammi di Bode per il ponte di Wien invertito \phantom {trollolollo} (Notch).}
    \label{fig:notch}
\end{wrapfigure}


In questo caso tutte le componenti circuitali sono note ($R1=R2=R3=R4=1 \si{\kilo\ohm}$ e $C1=C2=0.1 \si{\micro\farad}$) e ciò che faremo sarà misurare il voltaggio in CH1 e CH2 e rispettiva differenza di fase per diversi valori di frequenza $\nu$. Riportiamo ora le equazioni del circuito ricavate analiticamente: 

\noindent
\begin{minipage}{.65\linewidth}
\begin{equation}
V(CH1) = (\frac{V_{in}}{3+i(\omega R C - \frac{1}{\omega R C})})
\label{eq:V1}
\end{equation}
\end{minipage}%
\begin{minipage}{.5\linewidth}
\begin{equation}
V(CH2) = \frac{V_{in}}{2}
\label{eq:V(CH2)}
\end{equation}
\end{minipage}


\begin{equation}
\varphi_{2|1}=arctan[\frac{1}{3 C R \omega}-\frac{C R \omega}{3}]
\label{eq:F12}
\end{equation}


Poichè a noi interessa il valore di tensione che il circuito attaccato al filtro sente, dovremo fare $V_{out}=V(CH1)-V(CH2)$, da cui segue immediatamente:

\begin{equation}
V_{out}=\frac{1}{2} \sqrt{1-\frac{8 C^2 R^2 \omega^2}{C^4 R^4 \omega^4+7 C^2 R^2 \omega^2+1}}
\end{equation}

%usare questa eq nello script python !! phi=arctan[1/2 Sqrt[1 - (8 C^2 R^2 x^2)/(1 + 7 C^2 R^2 x^2 + C^4 R^4 x^4)]]


Come vediamo subito, sia per $\omega \rightarrow \infty$ che per $\omega \rightarrow 0$, $V_{out}=\frac{1}{2}$. Eseguendo una semplice derviata e ponendola uguale a zero, si verifica banalmente che il valore della frequenza di risonanza è $\nu_0=\frac{1}{2 \pi R C}$.

Per l'acquisizione dati abbiamo deciso di prendere i dati circa ogni 5 gradi di sfasamento tra i segnali. Ciò è stato fatto sapendo che la scala più opportuna per l'analisi è quella logaritmica su tutti i dati tranne che sugli angoli. \`E dunque una scelta saggia basarsi sugli angoli per decidere un metodo di lavoro.


I dati rilevati sono riportati nel grafico in Fig \ref{fig:notch}. Come vediamo nel primo grafico, il valore per $\nu=0$ è circa $-6 \si{\decibel}$. Imponendo la condizione $-6=20Log_{10}(\frac{V_{out}}{V_{in}})$, si ottiene subito la relazione $V_{out} \approx \frac{1}{2} V_{in}$. Notiamo come l'attenuazione di segnale non superi i $16 \si{\decibel}$. Se lo paragoniamo al filtro notch costruito utilizzando condensatore ed induttanza vediamo subito che il ponte di Wien non è così efficace (eravamo infatti arrivati a -30/-40dB). 

Inoltre notiamo tuttavia che la fase non è quella tipica del filtro notch già studiato nella precedente esperienza. Questo è dovuto al fatto che noi abbiamo considerato la differenza di fase tra i due segnali ai capi del ponte di Wien, non tra segnale in input e differenza dei due segnali in output. Tale relazione risulta infatti essere :

\begin{equation}
\varphi_{out|in}=arctan[\frac{2 C R \omega \left(C^2 R^2 \omega^2-1\right)}{C^4 R^4 \omega^4+C^2 R^2 \omega^2+1}]
\end{equation}

\pagebreak

Come vediamo da Fig. 468, la forma di tale equazione risulta proprio essere quella di un filtro notch.

% questa dovrebbe essere la fase in tipo notch phi=arctan[(2 C R x (-1 + C^2 R^2 x^2))/(1 + C^2 R^2 x^2 + C^4 R^4 x^4)]




