\section{Conclusioni}
Siamo riusciti a stimare una corrente di attivazione per l'interruttore differenziale. Tale valore di corrente efficace risulta ovviamente inferiore al valore dichiarato, in quanto il costruttore deve assolutamente essere certo che il sistema si attivi entro i $30\si{\milli\ampere}$. Per quanto riguarda i tempi di intervento abbiamo osservato valori molto diversi. A volte sembra anche che lo strumento non si sia attivato durante il primo periodo (tempi superiori a $20\si{\milli\second}$, il periodo del segnale sinusoidale). Non sappiamo perchè ciò sia avvenuto ma siamo riusciti a stimare attraverso il procedimento sopra trattato un limite inferiore al tempo di attivazione. Infatti dai nostri calcoli risulta che anche nella situazione ''migliore`` il tempo di intervento non potrà essere inferiore a $10\si{\milli\second}$. Poichè l'interruttore differenziale funziona solo se vi è una perdita di corrente, si consiglia di togliere la corrente da casa prima di cambiare una lampadina.

T

E

T

T

E

Nel grafico il t-att deve essere cambiato con tau mentre il t-att vero deve partire dallo zero...