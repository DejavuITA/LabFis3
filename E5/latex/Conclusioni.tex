\section{Conclusioni}
Siamo riusciti a stimare una corrente di attivazione per l'interruttore differenziale. Tale valore di corrente efficace risulta ovviamente inferiore al valore dichiarato, in quanto il costruttore deve assolutamente essere certo che il sistema si attivi entro i $30\si{\milli\ampere}$.

Per quanto riguarda i tempi di intervento abbiamo osservato valori molto diversi. A volte sembra anche che lo strumento non si sia attivato durante il primo periodo (tempi superiori a $20\si{\milli\second}$, il periodo del segnale sinusoidale). Ciò può essere dovuto al fatto che l'interruttore differenziale è progettato per lavorare ad una d.d.p. di $\SI{220}{\volt}_{rms}$ e non a $\SI{7.5}{\volt}_{rms}$ come nel nostro caso. Siamo comunque riusciti a stimare attraverso il procedimento sopra trattato un limite inferiore al tempo di attivazione. Infatti dai nostri calcoli risulta che anche nella situazione ``migliore'' il tempo di intervento non potrà essere inferiore a $10\si{\milli\second}$. %Poichè l'interruttore differenziale funziona solo se vi è una perdita di corrente, si consiglia di togliere la corrente da casa prima di cambiare una lampadina.\\

%/$\quad\,\,\,\,$ si consiglia di non mettere le dita nella presa indossando delle calzature particolarmente isolanti.
% mi è piaciuta particolarmente la cosa della lampadina, però non è tecnicamente la cosa migliore da dire: ora sono indeciso se lasciare quella per la spondaneità oppure cambiarla con qualcosa di tecnicamente corretto :/