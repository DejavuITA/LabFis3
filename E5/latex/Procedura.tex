\section{Stima della corrente di attivazione}
L'interruttore differenziale sfrutta l'induzione magnetica per rilevare perdite di corrente e interviene aprendo il circuito. Esso ha al suo interno due bobine avvolte in modo che, quando il circuito funziona correttamente, ovvero tanta corrente si ha in entrata quanta in uscita, i campi magnetici generati dalle bobine si annullano a vicenda. Quando si ha invece una perdita di corrente (ad esempio a terra), i campi magnetici generati sono diversi e si ha dunque un effetto di induzione elettromagnetica su una terza bobina che, attivando un meccanismo, apre il circuito. Come mostrato in Fig. (cazzi), abbiamo montato il circuito facendo passare un solo cavo nell'interruttore differenziale. Così facendo, abbiamo simulato il caso in cui la perdita di corrente è del 100\% a terra. Misurando poi, attraverso il multimetro, per quale valore di corrente l'interruttore scattava e ripetendo le misure varie volte, abbiamo stimato un valore di corrente di attivazione. 

Avendo alimentato il circuito con una tensione alternata a $7.5V_{rms}$, abbiamo utilizzato una decade di resistenze per poter ottenere diversi valori di corrente nel circuito. Sono stati eseguiti 15 campionamenti, con valor medio di $(26.5\pm0.3)\si{\milli\ampere}$. Abbiamo osservato che il valore di corrente di attivazione si aveva per un valore di resistenza di circa $300\si{\ohm}$. 


\section{Stima del tempo di intervento}
Per stimare il tempo di intervento dell'interruttore abbiamo utilizzato l'oscilloscopio in modalità acquisizione singola. In questa modalità lo strumento inizia a prendere dati solo nel momento in cui rileva un segnale, bloccandoli sullo schermo. Risulta così possibile osservare un fenomeno isolato e non periodico bloccandone l'immagine e analizzandola successivamente. Come mostrato in Fig.(fighe), sono stati collegati i canali dell'oscilloscopio in modo da avere sullo schermo sempre il segnale ai capi del generatore di tensione e quello ai capi dell'interruttore differenziale. Così facendo, fino a quando l'interruttore differenziale rimane chiuso, la ddp ai suoi capi sarà nulla mentre dopo l'apertura osserveremo un andamento identico a quello in input. Per stimare il tempo di intervento abbiamo spento il generatore di tensione