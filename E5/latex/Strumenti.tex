Verificare il funzionamento di un interruttore differenziale misurandone la corrente di attivazione e stimandone il tempo di intervento.

\section{Strumenti}
%
\phantom{porcodio!}
\noindent
\begin{minipage}{.5\linewidth}
$\bullet \quad$Generatore di tensione alternata ($\SI{7.5}{\volt}_{rms}$)\\
$\bullet \quad$Cablaggio\\
%$\bullet \quad$Resistenze da \SI{1}{\kilo\ohm} e \SI{10}{\kilo\ohm}, condensatori da \SI{0.1}{\micro\farad}\\
%$\bullet \quad$Condensatore di capacità incognita\\
$\bullet \quad$Decade di resistenze\\
$\bullet \quad$Interruttore differenziale \emph{Schneider}
\end{minipage}%
\begin{minipage}{.5\linewidth}
$\bullet \quad$Oscilloscopio (Agilent DSO-X 2002A)\\
\phantom{xxxx}\SI{70}{\mega\hertz} bandwidth, 2GSa/s sample rate\\
$\bullet \quad$Multimetro digitale (Agilent 34410A)\\
%$\bullet \quad$Generatore di forme d'onda (Agilent 33120A)\\
%\phantom{xxxx}frequency range: \SI{100}{\micro\hertz} - \SI{15}{\mega\hertz}\\
\end{minipage}